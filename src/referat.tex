\chapter*{РЕФЕРАТ}
\addcontentsline{toc}{chapter}{РЕФЕРАТ}

Расчетно-пояснительная записка 20~с., 7~рис., 10 источников, 1 прил.
\begin{sloppypar}
\MakeUppercase{компьютерное зрение, автопилот, дорожные знаки, поиск контуров, метод сравнения с эталоном, сверточные нейронные сети}
\end{sloppypar}

Объектом исследования являются методы обнаружения дорожных знаков на изображении.

Цель работы — рассмотрение технологии компьютерного зрения, в частности, для решения задачи
распознавания дорожных знаков на снимке.

В процессе работы были изучены существующие методы детектирования объектов на изображении и возможность их применения в системах с автопилотом.

\setcounter{page}{3}