\chapter{Анализ предметной области}

\section{Используемые библиотеки и платформы}
Задача распознавания объектов на снимке решается с применением технологии компьютерного зрения. Существует ряд программных платформ, которые позволяют реализовывать алгоритмы для решения данной задачи~\cite{comp_vision}. Ниже представлены некоторые из них:
\begin{itemize}[label=---]
    \item OpenCV --- один из самых популярных пакетов, который распространяется бесплатно и имеет открытый исходный код. Это библиотека, реализованная на C++, но имеющая API для таких популярных языков программирования, как Python, Java, Matlab и других.
    \item PCL --- тоже открытая платформа, в которой реализованы алгоритмы разных методов для обработки двумерных и трехмерных изображений.
    \item ROS --- специальная библиотека, разработанная для управления робототехникой.
    \item CUDA --- пакет, разработанный компанией \textit{NVIDIA}, который нацелен на ускорение работы системы, обрабатывающей изображения, путем параллельных вычислений на графических процессорах этой компании.
\end{itemize}

\section{Компьютерное зрение в области транспортных средств}

Как было сказано выше, машинное обучение применяется во многих сферах жизни человека, и область транспортных средств не является исключением. Компьютерное зрение активно используется при разработке новых систем управления автомобилем и помощи водителю. 

В связи с бурным ростом производства беспилотных автомобилей~\cite{tadviser} создается необходимость в развитии и улучшении технологий автопилота. В частности, они дали толчок для эволюции методов обработки изображений с использованием нейронных сетей (рисунок \ref{img:example3}~\cite{neuro_sys}). Сложная структура сетей позволяет выполнять операции нелинейного преобразования, что в результате делает возможным решение задачи распознавания дорожных знаков.

\imgScale{1}{example3}{Пример детектирования дорожных знаков с помощью нейронных сетей}



