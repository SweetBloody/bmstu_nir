\chapter*{ЗАКЛЮЧЕНИЕ}
\addcontentsline{toc}{chapter}{ЗАКЛЮЧЕНИЕ}

Цель, которая была поставлена в начале научно-исследовательской работы, была достигнута: изучена технология компьютерного зрения и методы, применяемые для решения задачи распознавания дорожных знаков на снимке.

Решены все поставленные задачи:

\begin{itemize}[label=---]
	\item проведено исследование существующих методов распознавания объектов на снимке;
	\item определены преимущества и недостатки рассмотренных методов;
    \item проанализирована возможность применения изученных алгоритмов компьютерного зрения для распознавания дорожных знаков на снимке системами с автопилотом.
\end{itemize}

В ходе исследования были определены особенности, преимущества и недостатки рассмотренных подходов к распознаванию объектов на изображении.

Алгоритмы поиска контуров являются простыми в реализации с точки зрения математики, однако они имеют существенные недостатки и ограничения:
\begin{itemize}[label=---]
	\item они не дают нужного результата на изображених с плохим цветовым контрастом, так как становится невозможным выделение контура;
	\item они не способны обработать перекрытие одних объектов другими;
    \item большинство таких методов неусточивы к шуму.
\end{itemize}

Методы сравнения с эталоном удобно применять, когда имеются заготовленные шаблоны обнаруживаемых объектов разделенные по классам. Одной из главных сложностей такого подхода является правильное определение используемой меры близости исследуемого экземпляра и эталона, от которой зависят результаты работы алгоритма.

Сверточные нейронные сети возможно использовать в случае, когда имеется большой объем обучающих данных. Если же количество тренировочных данных невелико, то возникает одна из главных проблем нейронных сетей --- переобучение. В тех случаях, в которых другие методы сталкиваются с ограничениями, связанными с качеством изображения, такая модель справляется лучше за счет свойства самообучения.

Стоит отметить, что алгоритмы компьютерного зрения и обработки изображений постоянно оптимизируются и улучшаются с целью повышения качества результата, и вполне возможно, что скоро системы с автопилотом станут обыденностью.