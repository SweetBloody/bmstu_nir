\chapter*{ВВЕДЕНИЕ}
\addcontentsline{toc}{chapter}{ВВЕДЕНИЕ}

В современном мире люди все чаще слышат об искусственном интеллекте из самых разных источников и в различных сферах деятельности. Технологии нейронных сетей, машинного обучения и глубокого обучения стремительно развиваются и используются человеком во многих отраслях его работы, например:
\begin{itemize}[label=---]
    \item медицине и фармацевтике;
    \item транспортных системах с контролем нагрузки;
    \item системах <<умный дом>>;
    \item системах автопилота;
    \item аналитике данных;
    \item распознавании речи и рукописного текста;
    \item цифровых помощниках.
\end{itemize}

Целью же данной научно-исследовательской работы является рассмотрение технологии компьютерного зрения, в частности, для решения задачи распознавания дорожных знаков на снимке.

Актуальность этой темы объясняется повышением интереса к системам автопилота. По данным исследования, проведенным \textit{Gartner}, мировой рынок беспилотных автомобилей стремительно растет. Оказалось, что в 2018 году общее количество новых полностью автоматизированных транспортных средств составило 137 129 единиц, а в 2019 году —-- 332 932 единиц. По данным аналитиков, к 2023 году количество самоуправляемых машин достигнет 745 705 единиц~\cite{tadviser}.

Для достижения поставленной в работе цели предстоит решить следующие задачи:
\begin{itemize}[label=---]
    \item провести исследование существующих методов распознавания объектов на снимке;
    \item определить преимущества и недостатки рассмотренных методов;
    \item проанализировать возможность применения изученных алгоритмов компьютерного зрения для распознавания дорожных знаков на снимке системами с автопилотом.
\end{itemize}


