\addcontentsline{toc}{chapter}{СПИСОК ИСПОЛЬЗОВАННЫХ ИСТОЧНИКОВ}

\renewcommand\bibname{СПИСОК ИСПОЛЬЗОВАННЫХ ИСТОЧНИКОВ}
\begin{thebibliography}{}

\bibitem{tadviser} Беспилотные автомобили (мировой рынок) [Электронный ресурс]. --- URL: \url{https://www.tadviser.ru/index.php/%D0%A1%D1%82%D0%B0%D1%82%D1%8C%D1%8F:%D0%91%D0%B5%D1%81%D0%BF%D0%B8%D0%BB%D0%BE%D1%82%D0%BD%D1%8B%D0%B5_%D0%B0%D0%B2%D1%82%D0%BE%D0%BC%D0%BE%D0%B1%D0%B8%D0%BB%D0%B8_(%D0%BC%D0%B8%D1%80%D0%BE%D0%B2%D0%BE%D0%B9_%D1%80%D1%8B%D0%BD%D0%BE%D0%BA)} (дата обращения: 02.12.2022)

\bibitem{comp_vision} Горячкин Б.С., Китов М.А. Компьютерное зрение --- Саранск: E-Scio, 2020.
%--- URL: \url{https://cyberleninka.ru/article/n/kompyuternoe-zrenie-1} (дата обращения: 26.11.2022).

\bibitem{neuro_sys} Девяткин А.В., Филатов Д.М. Нейросетевая система обнаружения знаков дорожного движения --- СПб.: СПбГЭТУ <<ЛЭТИ>>, 2019.
%--- URL: \url{https://scm.etu.ru/assets/files/2019/scm2019/papers/3/189.pdf} (дата обращения: 27.11.2022)

\bibitem{analysis_img} Калинина Н.Д., Куров А.В. Анализ методов распознавания и поиска образов на космических снимках --- М.: Издательство МГТУ им. Н.Э. Баумана, 2012.

\bibitem{border} Сакович И.О., Белов Ю.С. Обзор основных методов контурного анализа для выделения контуров движущихся объектов. --- М.: Издательство МГТУ им. Н.Э. Баумана, 2014.
%\url{https://cyberleninka.ru/article/n/obzor-osnovnyh-metodov-konturnogo-analiza-dlya-vydeleniya-konturov-dvizhuschihsya-obektov} (дата обращения: 10.12.2022)

\bibitem{etalon} Чичварин Н.В. Распознавание образов. --- М.: Издательство МГТУ им. Н.Э. Баумана, 2016.
%--- URL: \url{https://ru.bmstu.wiki/%D0%A0%D0%B0%D1%81%D0%BF%D0%BE%D0%B7%D0%BD%D0%B0%D0%B2%D0%B0%D0%BD%D0%B8%D0%B5_%D0%BE%D0%B1%D1%80%D0%B0%D0%B7%D0%BE%D0%B2#cite_note-1} (дата обращения: 10.12.2022)

\bibitem{deep_learning} Шолле Ф. Глубокое обучение на Python --- CПб.: Питер, 2018. --- 400с.

\bibitem{znak_cv} Медведев М.В., Кирпичников А.П., Синичкина Т.А. Детектирование дорожных знаков при помощи компьютерного зрения. --- Казань: Вестник Казанского технологического университета, 2016.
%\url{https://cyberleninka.ru/article/n/detektirovanie-dorozhnyh-znakov-pri-pomoschi-kompyuternogo-zreniya} (дата обращения: 18.12.2022).

\bibitem{some_inf} Якимов П.Ю. Распознавание дорожных знаков в реальном времени с использованием мобильного ГПУ --- Самара: Издательство СГАУ им. академика С.П. Королева, 2016.

\bibitem{neural_struct} Каковкин П.А., Друки А.А., Спицын В.Г. Применение алгоритмов глубокого обучения для локализации и распознавания дорожных знаков на снимке --- Томск: Издательство ТПУ, 2015.

\end{thebibliography}


